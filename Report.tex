%   This is a template for a senoir honours project report.
\documentclass[a4paper,12pt]{article}
\usepackage{fullpage}
\usepackage[pdftex]{graphicx}

%
%                       This section generates a title page
%                       Edit only the sections indicated to put
%                       in the project title, your name, supervisor,
%                       project length in weeks and submission date
%
\begin{document}
\pagestyle{empty}                       % No numbers on title page      

\par\noindent\includegraphics[width=12cm]{PandA_crest.pdf}

\par\noindent                                           % Centre Title, and name
\vspace*{2cm}
\begin{center}
        \Large\bf \Large\bf Senior Honours Project\\
        \LARGE\bf DNA Unzipping and Overstretching
\end{center}
\vspace*{0.5cm}
\begin{center}
        \bf Ben Henderson\\                               % Replace with your name and
        October 2021                                   % Submission Date
\end{center}
\vspace*{5mm}
%
%                       Insert your abstract HERE                
\begin{abstract}
        The abstract is a short, concise explanation of the project
        covering the aims, outlines of techniques used and a short
        summary of the results. It should contain enough information to
        make the aims and success of the project clear, but contain no details.
        A typical abstract should be between 50 and 100 words.
\end{abstract}

\vspace*{1cm}

\subsubsection*{Declaration}

\begin{quotation}
  I declare that this project and report is my own work.
\end{quotation}

\vspace*{2cm}
Signature:\hspace*{8cm}Date:  31/10/2021         % You can include a scanned version 
                                                 % of you signature here, or comment 
                                                 % this line out

\vfill
{\bf Supervisor:} Dr. D. Marenduzzo                 % Change to suit
\hfill
10 Weeks                                         % Change to suit
\newpage
%
%                       End of Title Page
\pagestyle{plain}                               % Page numbers at bottom
\setcounter{page}{1}                            % Set page number to 1
\tableofcontents                                % Makes Table of Contents
\section{Introduction}

The introduction section of the report should start by setting out the
motivation of the project, the aims, and an outline of techniques used.
It should also incorporate the literature review. This section should 
cover the theory of the material in the project in sufficient detail to 
make the following work understandable to the average physicist. What has 
been done before in this area, what different techniques have been 
employed? It should not contain large sections of standard
bookwork, but should contain references to this material. It should also 
contain references to similar work in the same field to put your work in 
the correct context.

The structure of the report is flexible, and will depend on the details
of the project being undertaken. For example, you might want to have a 
separate overall introduction section, and then a more specific background section 
on a method or approach. It should only contain relevant information. For example, 
a life history of the inventor of the technique to be used is
totally irrelevant\footnote{I have seen a report that contained three pages
on the life of Gabor, and it was not very interesting.}. Here use common sense
and the general rule, ``If in doubt: leave it out'', however
include information that you judge would be useful to one of your
peers if they were to repeat the project.

An introduction often ends with a sentence or two which summarises the main
results which are to come, setting the scene of the rest of the report. 




\section{Methods}

This section should contain the details of the method employed. 
As in the previous sections, standard techniques should not be written
out in detail. For example, if you use an oscilloscope to take a
measurement, the theory of the CRO tube\footnote{Don't laugh, I have actually
seen this.} is {\bf not relevant}. In computational projects this
section should be used to explain the algorithms used or details of any specific 
software. Long detailed sections of theory, data tables
and details of computational code used in data analysis only should not
appear in this section, but may be included in the appendices if relevant. You can
include unnumbered equations
\[
\frac{dy}{dx}=2x-y^2
\]
as well as numbered equations
\begin{equation} \label{myfirstequation}
\mathbf{\nabla}\times\mathbf{E} = -\frac{\partial \mathbf{B}}{\partial t}
\end{equation}
which can then be referred to later [as in Eq.~(\ref{myfirstequation})].

This section should emphasise the philosophy of the approach used
and detail novel techniques. However
please note: this section in {\bf not} a blow-by-blow account of what
you did throughout the project, and in particular it should {\bf not} 
contain large detailed sections about things you tried and found to be
completely wrong. Remember you are writing a technical report, and
not a diary. If however you find that a technique that was expected to
work failed, that is a valid result and should be included. You should 
also try to be clear where you are using a standard technique/piece of 
software, and where you have developed a new method or new code specifically 
for the project.

Here logical structure is particularly important, and you may find that
to maintain good structure you may have to present the experiments
in a different order from the one in which you carried them out.

Note that if the project consists of a series of short experiments
each of which requires a different theory and method, it may be appropriate
to have one {\bf Theory, Method, Results} section for each
experiment.


\section{Results \& Discussion}

This section should detail the obtained results in a clear,
easy-to-follow manner. Remember long tables of numbers are just as boring to
read as they are to write. Use graphs to present your results wherever 
practicable. Plots should be numbered and referred to in the text. 
The distribution of angles for the $L=100$ case are plotted in Figure 8.
When quoting results or measurements do not forget about units and errors. 
Remember there are two basic types of errors, random and systematic, 
which you must consider. Remember also the difference between an error 
and a mistake, computer program bugs are mistakes.
 
%
%                       Here is how to inserted a centered
%                       pdf file
%
\begin{figure}[htb]     %Insert a figure as soon as possible
        \begin{center}
        \includegraphics[width=7cm]{example_plot.pdf}
        	\end{center}
	\caption{This is an inserted figure from a pdf file. It is not a particularly good example, in that the numbers are small and difficult to read, axes labels are too small or missing and do not have units, and there are multiple lines which are not explained. A caption should be provided for each figure which explains what the elements of the plot are (lines, points, errorbars). If multiple lines or sets of points are shown a legend can also be provided. Multiple related plots can be combined as sub-panels.}
\end{figure}

Again be selective in what you include. Half a dozen
tables that contain totally wrong data you collected while you forgot
to switch on the power supply are {\bf not relevant} and will frequently
mask the correct results. It may be more appropriate to show one or two 
plots with examples of ``typical'' behaviour, rather than many plots which 
look identical.

This section also contains a discussion of the results. This should
include a discussion of the experimental and/or numerical errors, and a
comparison with the predictions of the background and theory underlying
the techniques used. It should highlight particular strengths
and/or weaknesses of the methods used.

 
\section{Conclusion}

Again, the report structure is flexible, and you might want to have separate
results, discussion and conclusion sections, or you might want to combine
the discussion with either the results or the conclusions. In any case, the 
conclusions section should summarise the results obtained, detail
conclusions reached, suggest future work, and and changes that you would make 
if you repeated the experiment. 
If you have opted to have multiple {\bf Theory, Method, Results}
sections, draw all the results together in a {\bf single} conclusion.

\section{References}

Don't forget to include a \textbf{references} section. Detail the relevant 
references which should be cited at the correct place in the text of the 
report. Citations can be done in passing, for example at the end of a 
sentence~\cite{mel1991}. Or can be explicit (for further details see 
Ref.~\cite{sindhikara2008}). There are no fixed rules as to how many 
references are needed. 
When you cite a reference you must give sufficient information, e.g., for a journal article give, {\it Author}, {\it Title of
article},
{\it Journal Name}, {\it Volume}, {\it Page}, and {\it Year}, 
while for a book give, {\it Author}, {\it Title},
{\it (Editor if there is one)}, {\it Publisher}, and {\it Year}.  
Being consistent is more important that the specific reference format
you use.

\begingroup
\renewcommand{\section}[2]{}  % supresses a further references title 

\begin{thebibliography}{10}
\bibitem{mel1991}
V.~I. Mel'nikov. ``The Kramers problem: Fifty years of development''.
\textit{Physics Reports} \textbf{209}, 1 (1991).

\bibitem{sindhikara2008}
D.~Sindhikara, Y.~Meng, and A.~E. Roitberg.
``Exchange frequency in replica exchange molecular dynamics''.
\textit{J. Chem. Phys.} \textbf{128}, 01B609 (2008).

\bibitem{russo2009}
J.~Russo, P.~Tartaglia, and F.~Sciortino.
``Reversible gels of patchy particles: role of the valence''.
\textit{J. Chem. Phys.} \textbf{131}, 014504 (2009).
\end{thebibliography}

\endgroup

% Now include appendices

\appendix
\section{Appendices}

Material that is useful background to the report, but is not essential,
or whose inclusion within the report  would detract from its
structure and readability, should be included in appendices. Typical
material could be diagrams of electronic circuits built, specialist
data tables used to analyse results, details of computer programs
written for analysis and display of results. Again be selective. The appendix is {\bf not} an excuse for you to add every
last detail and piece of data, but should be used to assist the reader
of the report by supplying additional material. Many reports will not require
appendices, and if the report is complete without the additional
material leave it out.

\end{document}
